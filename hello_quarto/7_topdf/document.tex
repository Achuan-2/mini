% Options for packages loaded elsewhere
\PassOptionsToPackage{unicode}{hyperref}
\PassOptionsToPackage{hyphens}{url}
\PassOptionsToPackage{dvipsnames,svgnames,x11names}{xcolor}
%
\documentclass[
  a4paper,
]{article}

\usepackage{amsmath,amssymb}
\usepackage{lmodern}
\usepackage{iftex}
\ifPDFTeX
  \usepackage[T1]{fontenc}
  \usepackage[utf8]{inputenc}
  \usepackage{textcomp} % provide euro and other symbols
\else % if luatex or xetex
  \usepackage{unicode-math}
  \defaultfontfeatures{Scale=MatchLowercase}
  \defaultfontfeatures[\rmfamily]{Ligatures=TeX,Scale=1}
\fi
% Use upquote if available, for straight quotes in verbatim environments
\IfFileExists{upquote.sty}{\usepackage{upquote}}{}
\IfFileExists{microtype.sty}{% use microtype if available
  \usepackage[]{microtype}
  \UseMicrotypeSet[protrusion]{basicmath} % disable protrusion for tt fonts
}{}
\makeatletter
\@ifundefined{KOMAClassName}{% if non-KOMA class
  \IfFileExists{parskip.sty}{%
    \usepackage{parskip}
  }{% else
    \setlength{\parindent}{0pt}
    \setlength{\parskip}{6pt plus 2pt minus 1pt}}
}{% if KOMA class
  \KOMAoptions{parskip=half}}
\makeatother
\usepackage{xcolor}
\usepackage[top=30mm,left=20mm,heightrounded]{geometry}
\setlength{\emergencystretch}{3em} % prevent overfull lines
\setcounter{secnumdepth}{5}
% Make \paragraph and \subparagraph free-standing
\ifx\paragraph\undefined\else
  \let\oldparagraph\paragraph
  \renewcommand{\paragraph}[1]{\oldparagraph{#1}\mbox{}}
\fi
\ifx\subparagraph\undefined\else
  \let\oldsubparagraph\subparagraph
  \renewcommand{\subparagraph}[1]{\oldsubparagraph{#1}\mbox{}}
\fi


\providecommand{\tightlist}{%
  \setlength{\itemsep}{0pt}\setlength{\parskip}{0pt}}\usepackage{longtable,booktabs,array}
\usepackage{calc} % for calculating minipage widths
% Correct order of tables after \paragraph or \subparagraph
\usepackage{etoolbox}
\makeatletter
\patchcmd\longtable{\par}{\if@noskipsec\mbox{}\fi\par}{}{}
\makeatother
% Allow footnotes in longtable head/foot
\IfFileExists{footnotehyper.sty}{\usepackage{footnotehyper}}{\usepackage{footnote}}
\makesavenoteenv{longtable}
\usepackage{graphicx}
\makeatletter
\def\maxwidth{\ifdim\Gin@nat@width>\linewidth\linewidth\else\Gin@nat@width\fi}
\def\maxheight{\ifdim\Gin@nat@height>\textheight\textheight\else\Gin@nat@height\fi}
\makeatother
% Scale images if necessary, so that they will not overflow the page
% margins by default, and it is still possible to overwrite the defaults
% using explicit options in \includegraphics[width, height, ...]{}
\setkeys{Gin}{width=\maxwidth,height=\maxheight,keepaspectratio}
% Set default figure placement to htbp
\makeatletter
\def\fps@figure{htbp}
\makeatother

\usepackage{ctex}    % CJK 包
\usepackage[backend=biber,style=gb7714-2015,gbnamefmt=lowercase]{biblatex}
\setCJKmainfont[BoldFont={方正大标宋简体}]{方正宋三简体}
\setmainfont{Times New Roman} %英文字體調整
\makeatletter
\makeatother
\makeatletter
\makeatother
\makeatletter
\@ifpackageloaded{caption}{}{\usepackage{caption}}
\AtBeginDocument{%
\ifdefined\contentsname
  \renewcommand*\contentsname{Table of contents}
\else
  \newcommand\contentsname{Table of contents}
\fi
\ifdefined\listfigurename
  \renewcommand*\listfigurename{List of Figures}
\else
  \newcommand\listfigurename{List of Figures}
\fi
\ifdefined\listtablename
  \renewcommand*\listtablename{List of Tables}
\else
  \newcommand\listtablename{List of Tables}
\fi
\ifdefined\figurename
  \renewcommand*\figurename{图}
\else
  \newcommand\figurename{图}
\fi
\ifdefined\tablename
  \renewcommand*\tablename{表}
\else
  \newcommand\tablename{表}
\fi
}
\@ifpackageloaded{float}{}{\usepackage{float}}
\floatstyle{ruled}
\@ifundefined{c@chapter}{\newfloat{codelisting}{h}{lop}}{\newfloat{codelisting}{h}{lop}[chapter]}
\floatname{codelisting}{Listing}
\newcommand*\listoflistings{\listof{codelisting}{List of Listings}}
\makeatother
\makeatletter
\@ifpackageloaded{caption}{}{\usepackage{caption}}
\@ifpackageloaded{subcaption}{}{\usepackage{subcaption}}
\makeatother
\makeatletter
\@ifpackageloaded{tcolorbox}{}{\usepackage[many]{tcolorbox}}
\makeatother
\makeatletter
\@ifundefined{shadecolor}{\definecolor{shadecolor}{HTML}{f1f3f5}}
\makeatother
\makeatletter
\makeatother
\ifLuaTeX
  \usepackage{selnolig}  % disable illegal ligatures
\fi
\usepackage[]{biblatex}
\addbibresource{references.bib}
\IfFileExists{bookmark.sty}{\usepackage{bookmark}}{\usepackage{hyperref}}
\IfFileExists{xurl.sty}{\usepackage{xurl}}{} % add URL line breaks if available
\urlstyle{same} % disable monospaced font for URLs
\hypersetup{
  pdftitle={My Document},
  pdfauthor={苏济雄},
  colorlinks=true,
  linkcolor={blue},
  filecolor={Maroon},
  citecolor={Blue},
  urlcolor={Blue},
  pdfcreator={LaTeX via pandoc}}

\title{My Document}
\author{苏济雄}
\date{}

\begin{document}
\maketitle
\ifdefined\Shaded\renewenvironment{Shaded}{\begin{tcolorbox}[boxrule=0pt, enhanced, colback={shadecolor}, frame hidden, breakable]}{\end{tcolorbox}}\fi

\renewcommand*\contentsname{Contents}
{
\hypersetup{linkcolor=}
\setcounter{tocdepth}{2}
\tableofcontents
}
\hypertarget{crossrefs}{%
\section{Crossrefs}\label{crossrefs}}

hello \cite{sevastre2022intracellular} hhh \cite{kucera2020prostate}
测试中文 \cite{董文鸳2011我国谷歌学术搜索研究综述}

\hypertarget{ux5965ux5361ux59c6ux5243ux5200ux539fux7406}{%
\section{奥卡姆剃刀原理}\label{ux5965ux5361ux59c6ux5243ux5200ux539fux7406}}

奥卡姆剃刀原理(Occam's Razor, Ockham's
Razor)又称``奥康的剃刀'',它是由 14
世纪英格兰的逻辑学家、圣方济各会修士奥卡姆的威廉(William of Occam,约
1285 年至 1349
年)提出。这个原理称为``\textbf{如无必要,勿增实体}'',即``简单有效原理''。正如他在《箴言书注》2
卷 15 题说``切勿浪费较多东西去做,用较少的东西,同样可以做好的事情。''

\hypertarget{ux5982ux4f55ux6446ux8131ux793eux4ea4ux81eaux5351}{%
\section{如何摆脱社交自卑}\label{ux5982ux4f55ux6446ux8131ux793eux4ea4ux81eaux5351}}

``很多人会好奇,为什么有的人能够在社交中这么自信地释放自己的魅力呢?
``------直到你认真地观察一朵花,它绽放的时候是完全展开的;它的花瓣、花蕊不管多么的脆弱------甚至是慌乱的,它都会把自己完全暴露在世人的面前。
``这就是它最美好的样子。

\hypertarget{ux6d4bux8bd5ux8868ux683c}{%
\section{测试表格}\label{ux6d4bux8bd5ux8868ux683c}}

\hypertarget{tbl-1}{}
\begin{longtable}[]{@{}cc@{}}
\caption{\label{tbl-1}Demonstration of pipe table syntax}\tabularnewline
\toprule()
header 1 & header 2 \\
\midrule()
\endfirsthead
\toprule()
header 1 & header 2 \\
\midrule()
\endhead
cell 1 & cell 2 \\
cell 3 & cell 4 \\
cell 5 & cell 6 \\
\bottomrule()
\end{longtable}

见 表~\ref{tbl-1}

\hypertarget{tbl-2}{}
\begin{longtable}[]{@{}cc@{}}
\caption{\label{tbl-2}Demonstration of pipe table syntax}\tabularnewline
\toprule()
header 1 & header 2 \\
\midrule()
\endfirsthead
\toprule()
header 1 & header 2 \\
\midrule()
\endhead
cell 1 & cell 2 \\
cell 3 & cell 4 \\
cell 5 & cell 6 \\
\bottomrule()
\end{longtable}

见 表~\ref{tbl-2}

\hypertarget{english}{%
\section{English}\label{english}}

When creating PDFs, you can choose to use either the default Pandoc
citation handling based on citeproc, or alternatively use natbib or
BibLaTeX. This can be controlled using the cite-method option. For
example: The default is to use citeproc (Pandoc's built in citation
processor).

See the main article on using Citations with Quarto for additional
details on citation syntax, available bibliography formats, etc.


\printbibliography[title=References]


\end{document}
